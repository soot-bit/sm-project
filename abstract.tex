% Abstracts are usually written in English, with a version in your
% mother tongue underneath
\chapter*{Abstract} 
\addcontentsline{toc}{chapter}{Abstract}
% Don't change anything above this.

A short description of your essay goes here. This should be between 100 and 300 words long. Around half a page saying what your problem or research question is, and how it is approached in this essay, and in some cases a conclusion. 

An abstract is more than a short summary of your completed research. It will tell the readers why they should be interested in your essay. If you do this well, it will make the readers want to read more. If you do this poorly, many people will stop reading at this point.

Typically, a good abstract will have a structure like this:

\begin{enumerate}
\item Motivation
\item Problem statement
\item Approach
\item Results
\item Conclusion
\end{enumerate}

You may wish to put the problem statement ahead of the motivation, depending on your topic, and it may not make sense to have a results section if your essay is a review of a particular field. You should be flexible when deciding how to structure your abstract. Of course, this is easiest if you do not start writing the abstract until you have completed a draft of the entire essay.

The writing style used in an abstract is different from the style used in the rest of your essay. You should be concise, clear, and direct. Avoid the use of extra words and get straight to the point. Try to fill your sentences with useful adjectives that convey as much relevant meaning as possible. The reader should understand why your essay is important, what makes it interesting, what your main results are, and (if applicable) what follows from your work. You may also wish to use key terms and phrases from your field because researchers who are looking for something relevant to their own work are more likely to search for these terms in abstracts.

Here is an example of a good abstract:

\begin{quote}
The diabetes epidemic among children in the United States constitutes one of the most serious health problems in the region. What makes the study and control of the disease even more complex is its interaction with the parallel obesity epidemic. This research provides a detailed review of a deterministic compartmental model for obesity and diabetes to understand better the dynamics of the twin epidemics in a typical American suburb. This model is built on traditional population models but also includes effects of diet and exercise. The population is studied using a differential equation that comes naturally from the model assumptions, using the method of linearization near the disease-free equilibrium and solved numerically with Python. Our results demonstrate that the prevalence of diabetes among youth is expected to significantly higher in a suburb than in the countryside or in an urban centre, mainly as a result of reduced exercise.
\end{quote}

Notice how the abstract begins with the motivation: the first sentence tells you exactly what the key issue is and the second sentence tells you why studying it is difficult. Next, the abstract clearly states the problem in a concise and direct sentence. There are no extra or ambiguous words but there is still enough detail to understand the key elements of the problem being studied. The following two sentences describe the approach used, again with simple and direct language. The final sentence is a statement of the overall conclusion, telling you what the main result is. After reading this abstract, one has a sense of what the essay is about, why it is interesting, what techniques were used, and what the chief outcome or result is.

Here is a poorly written abstract:

\begin{quote}
This essay examines the current developments of modelling crime rates using statistical methods. Regression models for time series data of crime instances are discussed in this essay. The statistical methods discussed are Empirical Orthogonal Functions (EOFs), Canonical Correlation Analysis (CCA) and Multivariate Regression (MVR). We examined the various statistical methods for modelling correlations by taking into account the variability of the data due to temporal and spatial proximity. Some of these methods include Maximum Likelihood Estimation (MLE), Generalized Method of Moments (GMM), and the use of the M-estimator. We considered both the Bayesian and the Frequentist approach to this problem using data collected from Essex, UK. The data, describing crime in a low income region near London, England, was collected over five years from 1985--1990.
\end{quote}

In this abstract, there is not much detail in each sentence and it is hard to understand exactly what was done in the essay. It is true that we get an idea of the techniques used -- but there is too much listing of methods and not enough of an explanation of the research problem. This abstract is boring and not very well structured and it gives the impression that the essay will not make for interesting reading. Also, notice that because there is no motivation and no conclusion in this abstract, it is hard to get an overall sense of the essay and what its key message is.

% At a unviersity like Stellenbosch you *must* produce an abstract in Afrikaans for your masters.
% At AIMS you are encouraged to repeat the abstract in your mother tongue
% French, Igbo, Mlagasy, etc. just write it using LaTeX's special
% characters.
% Arabic students use the arabtex package.
% Amharic use openoffice and export from there and import a figure here.
% Where the words do not exist put the English work in italics, or use mathematical symbols.


% Do not change anything below this except for adding your
% signature and name. And take the message to heart.
\vfill
\section*{Declaration}
I, the undersigned, hereby declare that the work contained in this essay is my original work, and that any work done by others or by myself previously has been acknowledged and referenced accordingly.

% Scan your signature into a small picture called 'signature.png' and insert it
% above your name and the date:
%\includegraphics[height=2cm]{signature.png} \newline \hrule
% Your name must be in English Capitalisation with no comma, and the Family name comes last. 
% Do note the date below. It is called the "deadline".
Firstname Middlename Lastname, 19 May 2011
