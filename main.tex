%-----------------------------------------------------------------------------
% Template for AIMS Structured Masters Research Project
%
% The fonts, linespacing, numbering, page styles, order
% of  Title/Abstract/TOC/Body/{Appendices}/Acknowledgements/References 
% are prescribed as the AIMS house style.
%
% Do not change them or add to it without getting approval first.
% Essays are not accepted if they do not follow house style.
% This is in preparation for your Masters where the university
% will be much more strict on the house style.
%



\documentclass{aimsessay}

%--------------------Preamble:
% \usepackage{etex} % fix for using xy and tikz in the same document
\usepackage[utf8]{inputenc}
\usepackage[T1]{fontenc}
\usepackage{lmodern}
\usepackage[round]{natbib}
\usepackage{amsmath}
\usepackage{amsfonts}
\usepackage{amssymb}
\usepackage{mathtools}
\usepackage{latexsym}
\usepackage{parskip}
\usepackage{enumerate}
\usepackage{tikz}
\usepackage{lipsum}
% Fix spacing before theorem environments
\makeatletter\def\thm@space@setup{%
  \thm@preskip=1.2\parskip \thm@postskip=0pt
  }
%\makeatother
% Reduce space before proof
%\makeatletter
\renewenvironment{proof}[1][\proofname]{\par
  \vspace{-\topsep}% remove the space after the theorem
  \pushQED{\qed}%
  \normalfont
  \topsep0pt \partopsep0pt % no space before
  \trivlist
  \item[\hskip\labelsep
        \itshape
    #1\@addpunct{.}]\ignorespaces
}{%
  \popQED\endtrivlist\@endpefalse
  \addvspace{6pt plus 6pt} % some space after
}
\makeatother
% Reduce spacing around section headings?
%
%-----------------------------------------------------------------------------
% To use external packages for specific needs, 
% get approval before adding them here (they
% should not override general AIMS house style and layout).
%
% Examples:
% 
% Load caption before arabtex
\usepackage{caption} %many figures in one figure (note subfigure and subfig are deprecated) 
\usepackage{subcaption} %many figures in one figure (note subfigure and subfig are deprecated) 
% For Arabic
%\usepackage{arabtex}
%\usepackage{utf8}
%\setcode{utf8}
% For tables:
\usepackage{booktabs} % \toprule, \midrule, \bottomrule
\usepackage{array}    % \newcolumntype
% 
% For figures:
%\usepackage[below,section]{placeins} % use \FloatBarrier in the body to really force a float somewhere. Please limit use.
\usepackage{float}  %"H" placement spec, for **really here** (i.e. not float)
%
% For code and algorithms
\usepackage{moreverb}   % \verbatimtabinput
% \usepackage{listings} % more flexible and complicated 
                        % than moreverb and algorithm
% 
% Others
% \usepackage[all]{xy}  %  if you use this uncomment \usepackage{etex} above to fix a conflict with tikz.
% \usepackage{sagetex}
% \usepackage{siunitx} % to typeset numbers, units, align decimals in tables.
% \usepackage{dcolumn} % less flexible but maybe faster than siunitx above.
% \usepackage{mathtools} % More maths, e.g. \mathclap.
%
% Others may be landscape, longtable, algorithm, algorithmic, etc.
% 
% ----------------------------------------------------------------------------
% An AIMS Essay can use the sectioning commands "\chapter", "\section",
% "\subsection". No "\subsubsection", "\paragraph", etc. They are disabled.
% 
% For Theorems and such, use the environments defined here:
% \begin{thm}...\end{thm} (or "lem", "defn", etc)
% 
% We put the number to the left of the Theorem heading.
\swapnumbers 
% 
% Theorems are in italics.
\theoremstyle{plain}
\newtheorem{thm}[subsection]{Theorem}
%
% Rest is not in italics.
\theoremstyle{definition} 
\newtheorem{lem}[subsection]{Lemma}
\newtheorem{cor}[subsection]{Corollary}
\newtheorem{conj}[subsection]{Conjecture}
\newtheorem{pro}[subsection]{Proposition}
\newtheorem{exa}[subsection]{Example}
\newtheorem{defn}[subsection]{Definition}
\newtheorem{rem}[subsection]{Remark}
% 
% If you have no theorems, but a lot of equations, maybe the
% following two lines are good. Beware of corresponding Equation
% and Example numbers though! Number equations by sections.
% 
\numberwithin{equation}{section}




%-----------------------------------------------------------------------------
 
% Your own command shortcuts can go here
% keep them clearly separate from other sections of the preamble
% It is good style to have only a few of these so that
% we can read one another's code. If you have to many, 
% then your code does not compile in someone else's template easily,
% and it makes it harder to read. These definitions are only
% meant for very often-used commands to save typing; Examples:
%
%\newcommand {\be}{\begin{equation}}
%\newcommand {\ee}{\end{equation}}
%\newcommand {\C}{\mathbb{C}} % Complex
%\newcommand {\Z}{\mathbb{Z}} % Integers
%\newcommand {\R}{\mathbb{R}} % Real
%\DeclareMathOperator{\sech}{sech} % declaring new math operators like \sin.
%  
%-----------------------------------------------------------------------------



% Title & Author
% On this page you must have NO full-word capitalizations, bold, or colour. 
% All AIMS research projects per year have the same title page.
% In English your family name is written last, i.e. Firstname LASTNAME
% English Capitalization, not as in some Francophone countries where
% you write LASTNAME, Firstname.
% Put your AIMS email address only please, for consistency,
% not gmail or some other webmail address.
\title{The Essay Title goes here Rafa\l Rafa\L{} test}
% Your name must be in English Capitalisation with no comma, 
% and the Family name comes last.
\author{Firstname Middlename Familyname (email@aims.ac.za)\\
% Then in the MAIN BODY use this:                  
%\begin{RLtext}
%نووووووسسسسح
%\end{RLtext}\\
%%%%%%%%%%%%%%%%%%%%%%%%%%%%%%%%%%%
%\begin{otherlanguage}{arabtext}
%شةشىغ
%\end{otherlanguage}\\
%%%%%%%%%%%%%%%%%%%%%%%%%%%%%%%%%%%%%
% Amharic students speak to me about how to add your name in your own alphabet.
% Everything here is prescribed; do not enter bold or ALL CAPS here,
% it will not be accepted.
African Institute for Mathematical Sciences (AIMS)\\
\\
% Example1
{\small Supervised by: Title Firstname Lastname}\\
{\small Institute of Supervisor, Country}%\\
% For second/more supervisors, continue with another line, e.g.
% {\small and Dr So And-so}
% {\small University of Life, Country}
% Don't put the department, it becomes too long.
}
\date{{\small 2 November 2023}\\%
  {\scriptsize\it Submitted in partial fulfillment of 
    a structured masters degree at AIMS South Africa}\\%
  \vspace{0.5cm}{\includegraphics{Images/AIMS_SA_Logo.pdf}}}





%-------------------------------------------------------------------------
\begin{document}
%\selectlanguage{english}
\pagestyle{empty}
\maketitle





% All other files are included via \input. 
% To compile in texmaker while viewing any of those
% without having to switch back, use
%   Options > Define Current Document as 'Master Document'
% To not have to close a PDF, remove viewpdf from quickbuild 
% and open the PDF (once) manually: it will auto-refresh or with control-r
% 
%-------------------------------------------------------------------------
% The abstract is the first thing we want to see. No acknowledgements or 
% dedications here. Fetch the abstract from a separate file.
% Please write it in English and in your mother tongue.
% An abstract should be less than half a page, so that both abstracts 
% (that is both languages) fit onto one page.
% We number roman numerals until the main body
\pagenumbering{roman}
% Abstracts are usually written in English, with a version in your
% mother tongue underneath
\chapter*{Abstract} 
\addcontentsline{toc}{chapter}{Abstract}
% Don't change anything above this.

A short description of your essay goes here. This should be between 100 and 300 words long. Around half a page saying what your problem or research question is, and how it is approached in this essay, and in some cases a conclusion. 

An abstract is more than a short summary of your completed research. It will tell the readers why they should be interested in your essay. If you do this well, it will make the readers want to read more. If you do this poorly, many people will stop reading at this point.

Typically, a good abstract will have a structure like this:

\begin{enumerate}
\item Motivation
\item Problem statement
\item Approach
\item Results
\item Conclusion
\end{enumerate}

You may wish to put the problem statement ahead of the motivation, depending on your topic, and it may not make sense to have a results section if your essay is a review of a particular field. You should be flexible when deciding how to structure your abstract. Of course, this is easiest if you do not start writing the abstract until you have completed a draft of the entire essay.

The writing style used in an abstract is different from the style used in the rest of your essay. You should be concise, clear, and direct. Avoid the use of extra words and get straight to the point. Try to fill your sentences with useful adjectives that convey as much relevant meaning as possible. The reader should understand why your essay is important, what makes it interesting, what your main results are, and (if applicable) what follows from your work. You may also wish to use key terms and phrases from your field because researchers who are looking for something relevant to their own work are more likely to search for these terms in abstracts.

Here is an example of a good abstract:

\begin{quote}
The diabetes epidemic among children in the United States constitutes one of the most serious health problems in the region. What makes the study and control of the disease even more complex is its interaction with the parallel obesity epidemic. This research provides a detailed review of a deterministic compartmental model for obesity and diabetes to understand better the dynamics of the twin epidemics in a typical American suburb. This model is built on traditional population models but also includes effects of diet and exercise. The population is studied using a differential equation that comes naturally from the model assumptions, using the method of linearization near the disease-free equilibrium and solved numerically with Python. Our results demonstrate that the prevalence of diabetes among youth is expected to significantly higher in a suburb than in the countryside or in an urban centre, mainly as a result of reduced exercise.
\end{quote}

Notice how the abstract begins with the motivation: the first sentence tells you exactly what the key issue is and the second sentence tells you why studying it is difficult. Next, the abstract clearly states the problem in a concise and direct sentence. There are no extra or ambiguous words but there is still enough detail to understand the key elements of the problem being studied. The following two sentences describe the approach used, again with simple and direct language. The final sentence is a statement of the overall conclusion, telling you what the main result is. After reading this abstract, one has a sense of what the essay is about, why it is interesting, what techniques were used, and what the chief outcome or result is.

Here is a poorly written abstract:

\begin{quote}
This essay examines the current developments of modelling crime rates using statistical methods. Regression models for time series data of crime instances are discussed in this essay. The statistical methods discussed are Empirical Orthogonal Functions (EOFs), Canonical Correlation Analysis (CCA) and Multivariate Regression (MVR). We examined the various statistical methods for modelling correlations by taking into account the variability of the data due to temporal and spatial proximity. Some of these methods include Maximum Likelihood Estimation (MLE), Generalized Method of Moments (GMM), and the use of the M-estimator. We considered both the Bayesian and the Frequentist approach to this problem using data collected from Essex, UK. The data, describing crime in a low income region near London, England, was collected over five years from 1985--1990.
\end{quote}

In this abstract, there is not much detail in each sentence and it is hard to understand exactly what was done in the essay. It is true that we get an idea of the techniques used -- but there is too much listing of methods and not enough of an explanation of the research problem. This abstract is boring and not very well structured and it gives the impression that the essay will not make for interesting reading. Also, notice that because there is no motivation and no conclusion in this abstract, it is hard to get an overall sense of the essay and what its key message is.

% At a unviersity like Stellenbosch you *must* produce an abstract in Afrikaans for your masters.
% At AIMS you are encouraged to repeat the abstract in your mother tongue
% French, Igbo, Mlagasy, etc. just write it using LaTeX's special
% characters.
% Arabic students use the arabtex package.
% Amharic use openoffice and export from there and import a figure here.
% Where the words do not exist put the English work in italics, or use mathematical symbols.


% Do not change anything below this except for adding your
% signature and name. And take the message to heart.
\vfill
\section*{Declaration}
I, the undersigned, hereby declare that the work contained in this essay is my original work, and that any work done by others or by myself previously has been acknowledged and referenced accordingly.

% Scan your signature into a small picture called 'signature.png' and insert it
% above your name and the date:
%\includegraphics[height=2cm]{signature.png} \newline \hrule
% Your name must be in English Capitalisation with no comma, and the Family name comes last. 
% Do note the date below. It is called the "deadline".
Firstname Middlename Lastname, 19 May 2011





% Don't go typing out the contents.
\tableofcontents
\newpage
% We switch to arabic numerals here where your page count starts
% Essays must be close to 30 pages long *starting here* and up to and including
% the conclcusion. It does not include the acknowledgements or references.
% 
% Figures may differ between topics, but they are not there
% to fill the pages -- they must add meaning.
% In general most figures should be 0.8 times the width of the page
% (perhaps wider in total when two or three columns of figures)
% See the example in chapter one for defining that. Be *consistent*
% in your presentation of information.
\pagenumbering{arabic}
\pagestyle{myheadings}




%-----------------------------------------------------------------------------
% Each chapter goes in a separate file
% Chapter titles are just examples
% Always have a question
% Note the Case Pattern used at AIMS
\chapter{Introduction}

\section{Problem Statement}
This is usually an introduction and problem statement. It can be called something more exciting and descriptive than ``Introduction'', perhaps? 

When you quote use the correct symbols as above for quotation marks!

This is where your essay starts. Please remember that it must be 25 pages from here on. This is the first paragraph. 

The is the second paragraph. Always move to the next paragraph by leaving an open line and never using the double backslash (\verb|\\|)
command. Those are reserved for arrays, tables, and so on.

Paragraphs are separated by blank lines in the \LaTeX\ code, and we set the line spacing, paragraph indentation,
and paragraph spacing in the preamble for you, according to AIMS house style.

\section{Moving On}
Let's look at bandwidth as in Fig. \ref{bandwidth} bit. Just to demonstrate a figure!

\begin{figure}[!h]
% Use "\centering" in floats (figure, table), but if you need to center
% some text (why?) use "\begin{center}...\end{center}".
\centering 
% Figure environments same as 0.8 * \textwidth please
% That does not necessarily mean the actual picture size,
% it is a guideline for the environment which could contain
% 2 or more pictures! Be consistent and follow the guidelines
% provided in your sources.
\includegraphics[width=0.8\textwidth]{images/bandwidth-colour.png}
\caption{Planning community bandwidth sharing costs. 
  Note caption capitalization.}
\label{bandwidth} 
% if you move the label it breaks the reference numbering; 
% always have it *after* the caption.
\end{figure}

\section{Bleah}
See Fig. \ref{bandwidth} to understand how figures work. It's OK to have figures on another page if you reference them correctly! if you use someone else's pictures, acknowledge them. And remember to check on the copyright. You can include tables in the same way but with the \verb|\begin{tabular}| and
 \verb|\end{tabular}| commands.

Remember how to include code with verbatim and to fix the tabs in python in a verbatim environment? It is by far best to have an include command for your code, not to re-edit it all the time!
\verbatimtabinput{code/mycode.py}

\section{Consistency}
Consistency is by far the most important thing to remember. Just after honesty and citing the work of others correctly as in \cite{AD92, Beardon}.
 % Introduction is usually a chapter itself.
\chapter{The Second Chapter}

When you get stuck, don't panic. 
The world is unlikely to end just now. 
Remember you can consult your supervisor, tutor, Nolu, Jan, 
and Jeff and Barry at agreed times. 

\begin{thm}[Jeff's Washing Theorem]
\label{thm:jwt}
If an item of clothing is too big, then washing it makes it bigger;
but if it is too small, washing it makes it smaller.
\end{thm}
\begin{proof}
Stated without proof. But a proof would look like this. 
\end{proof}

Notice that no Lemmas are required in the proof of Theorem \ref{thm:jwt}.

Use \textbackslash ref for tables, figures, theorems, etc. and \textbackslash eqref for equations.

Use \textbackslash ldots for continuation of commas $,\ldots,$ and \textbackslash cdots for continuation of operators $\times\cdots\times$.
 % Chapters might go from 2. problem statement, 
                 % through 3. model, to 4. analysis & results
\chapter{Third Chapter}

\lipsum[1]

\begin{thm}[My Theorem2]
This is my theorem2.
\end{thm}
\begin{proof}
And it has no proof2.
\end{proof}

Lorum ipsum.

\section{See?}

\lipsum[1]

\begin{thm}[My Theorem2]
This is my theorem2.
\end{thm}
\begin{proof}
And it has no proof2.
\end{proof}

\lipsum[1]

\begin{align} % do not use eqnarray. 
\label{2ya}
x & = y + y\\
\label{2yb}
& = 2y
\end{align}
Equations \eqref{2ya} and \eqref{2yb} are trivial.

\section{Numbering in AIMS essays}

Here is a conjecture:

\begin{conj}
The washing operation has fixed points.
\end{conj}

And here is an example:

\begin{exa}
5 Rand coin.
\end{exa}

\subsection{This is a subsection}

\lipsum[1]

\section{This is a section}

\lipsum[1]



 % You do not need to have exactly 4 chapters.
                 % It is probably a good minimum, with 5 chapters 
                 % average, and 7 chapters might be a maximum.
\chapter{The Second Squared Chapter}

An average research project may contain five chapters, but I didn't plan my work properly
and then ran out of time. I spent too much time positioning my figures and worrying
about my preferred typographic style, rather than just using what was provided.
I wasted days bolding section headings and using double slash line endings, and 
had to remove them all again. I spent sleepless nights configuring manually numbered lists
to use the \LaTeX\ environments because I didn't use them from the start or understand
how to search and replace easily with texmaker.

Everyone has to take some shortcuts
at some point to meet deadlines. Time did not allow to test model 
B as well. So I'll skip right ahead and put that under my Future Work section.


\section{This is a section} 

Some research projects may have 3, 5 or 6 chapters. This is just an example. 
More importantly, do you have at close to 30 pages?  
Luck has nothing to do with it. Use the techniques suggested for
writing your research project.

Now you're demonstrating pure talent and newly acquired skills. 
Perhaps some persistence. Definitely some inspiration. What was that about perspiration? 
Some team work helps, so every now and then why not browse your friends' research project and provide
some constructive feedback?

\subsection{Subsubsections are disabled}

Vivamus faucibus arcu ut cursus maximus. Aenean ac aliquet nulla. Duis efficitur varius malesuada. Etiam finibus risus et condimentum commodo. Mauris interdum ligula ut lacinia blandit. Curabitur commodo, mauris vel porttitor semper, ante risus pellentesque ipsum, non commodo sapien massa quis tortor. Vestibulum ante ipsum primis in faucibus orci luctus et ultrices posuere cubilia Curae; Phasellus ac massa commodo purus placerat pharetra id ut ex. Nam malesuada, turpis vel iaculis sodales, nisl ante fringilla tellus, et efficitur nisl felis at ligula. 
 % Conclusion is usually a chapter itself. 
%\input{chapter5} % You may have more chapters. (Use e.g. git add FILE)
% This is where we stop counting pages.
% Acknowledgements and References are not counted.





%-----------------------------------------------------------------------------
% See the acknowledgement.tex file and follow the instructions there.
\chapter*{Acknowledgements}
% Don't change anything above this.
% We do not number this or add it to the contents!
% Overly long acknowledgements are not professional.

I want to acknowledge AIMS and it's funders for supporting this work, as well as my supervisor, Prof OluwaBusayao Akinola from University of Ibadan.






%-----------------------------------------------------------------------------
% Note the errata page is not for now, it is for use during the examination.
% Not that you're going to have any errata.




%-----------------------------------------------------------------------------
% THE BIBLIOGRAPHY 
% Bibliography styles define how the bibliography is 
% listed and formatted. This is part of the AIMS house
% style and is only changed under exceptional circumstances
\renewcommand{\bibname}{References}
\nocite{*}
\bibliographystyle{myabbrvnat}
\bibliography{references}
\addcontentsline{toc}{chapter}{References}
%-----------------------------------------------------------------------------
\end{document}
